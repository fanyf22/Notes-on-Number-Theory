\chaptere{Group Extension}

%%%%%%%%%%%%%%%%%%%
%%% NEW SECTION %%%
%%%%%%%%%%%%%%%%%%%

\sectione{Group Extension and Second Cohomology Group}

\begin{analysis}
    Let $0\to A\xrightarrow iU\xrightarrow jG\to 0$ be a short exact sequence of groups, with $G$ finite and $A$ abelian. We take any elements $u_\sigma\in U$ for each $\sigma\in G$, such that $j(u_\tau)=\tau$. We define an action of $G$ over $A$ by $a^\sigma=u_\sigma au_\sigma^{-1}$. We claim that this action is well-defined, and independent on the choice of $u_\sigma$.

    Firstly, since $A$ is a normal subgroup of $U$, we see $u_\sigma au_\sigma^{-1}\in A$. Since $A$ is abelian, and distinct choices of $u_\sigma$ only differ by $A$, the choice of $u_\sigma$ does not affect the value of $a^\sigma$. Since $j(u_\sigma u_\tau)=j(u_{\sigma\tau})$, we have \[ a^{\sigma\tau}=u_{\sigma\tau}au_{\sigma\tau}^{-1}=u_\sigma u_\tau au_\tau^{-1}u_\sigma^{-1}=(a^\tau)^\sigma \] \[ a^\sigma b^\sigma=u_\sigma au_\sigma^{-1}\cdot u_\sigma bu_\sigma^{-1}=u_\sigma abu_\sigma^{-1}=(ab)^\sigma \] Thus it is indeed a group action. We call it the induced action by $U/A\approx G$.
\end{analysis}

\begin{definition}
    Let $A$ be a $G$-module. A \emph{group extension} of $A$ is a short exact sequence $0\to A\to E\to G\to 0$ (or $U/A\approx G$ for short) such that $G$ acts on $A$ by the induced action.
\end{definition}

\begin{analysis}
    In fact, we can describe $U$ by $A$ and a map $(\sigma,\tau)\mapsto a_{\sigma,\tau}$ explicitly. Given a group extension $U/A\approx G$, we take $a_{\sigma,\tau}=u_\sigma u_\tau u_{\sigma\tau}^{-1}$. First, each element in $U$ can be uniquely written in the form $au_\sigma$ for some $a\in A$ and $\sigma\in G$. Thus it suffices to tell its multiplication: \[ au_\sigma bu_\tau=ab^\sigma u_\sigma u_\tau=ab^\sigma a_{\sigma,\tau}u_{\sigma\tau} \] where $ab^\sigma a_{\sigma,\tau}\in A$ and $\sigma\tau\in G$. Thus we see a group extension of $A$ can be constructed by $U=A\times G$ as a set and $(a,\sigma)\cdot(b,\tau)=(ab^\sigma a_{\sigma,\tau},\sigma\tau)$ for some map $(\sigma,\tau)\mapsto a_{\sigma,\tau}$. Therefore, we wish to decribe what $a_{\sigma,\tau}$ induces a group extension, and what $a_{\sigma,\tau}$ induces the same group extension.

    Firstly, we try to give a condition of $a_{\sigma,\tau}$ to induce a group extension. We start from associative law: \[ ((a,\sigma)(b,\tau))(c,\gamma)=(a,\sigma)((b,\tau)(c,\gamma)) \] From the multiplication we obtained above, we see \begin{align*} ((a,\sigma)(b,\tau))(c,\gamma)&=(ab^\sigma a_{\sigma,\tau},\sigma\tau)(c,\gamma)=(ab^\sigma c^{\sigma\tau}a_{\sigma,\tau}a_{\sigma\tau,\gamma},\sigma\tau\gamma) \\ (a,\sigma)((b,\tau)(c,\gamma))&=(a,\sigma)(bc^\tau a_{\tau,\gamma},\tau\gamma)=(ab^\sigma c^{\sigma\tau}a_{\tau,\gamma}^\sigma a_{\sigma,\tau\gamma},\sigma\tau\gamma) \end{align*} Thus we see $a_{\sigma,\tau}a_{\sigma\tau,\gamma}=a_{\tau,\gamma}^\sigma a_{\sigma,\tau\gamma}$. Conversely, if this condition is satisfied, then we can define a multiplication on $U$ by $au_\sigma bu_\tau=ab^\sigma a_{\sigma,\tau}u_{\sigma\tau}$, and it is associative. The identity is $(a_{1,1}^{-1},1)$, and the inverse is $(b,\tau)^{-1}=(a_{1,1}^{-1}a_{\tau^{-1},\tau}^{-1}b^{-\tau^{-1}},\tau^{-1})$. In conclusion, the condition is $a_{\tau,\gamma}^\sigma=a_{\sigma,\tau}a_{\sigma\tau,\gamma}a_{\sigma,\tau\gamma}^{-1}$, or $\sigma a_{\tau,\gamma}=a_{\sigma,\tau}-a_{\sigma,\tau\gamma}+a_{\sigma\tau,\gamma}$ additively.
\end{analysis}

\begin{remark}
    Readers might take it for granted that $A\hookrightarrow U$ via the map $a\mapsto(a,1)$, but it is in fact not true, since $(a,1)\leftrightarrow au_1$. Thus, $a$ actually corresponds to $(au_1^{-1},1)=(aa_{1,1}^{-1},1)$.
\end{remark}

\begin{proposition} \label{thm:condition of cochain to induce group extension}
    Let $A$ be a $G$-module. Then the map $(\sigma,\tau)\mapsto a_{\sigma,\tau}$ induces a group extention if and only if $\sigma a_{\tau,\gamma}=a_{\sigma,\tau}-a_{\sigma,\tau\gamma}+a_{\sigma\tau,\gamma}$ for any $\sigma,\tau,\gamma\in G$.
\end{proposition}

\begin{lemma}
    If $a_{\sigma,\tau}$ induces a group extension, then $a_{1,\sigma}=a_{1,1}$ and $\sigma a_{\tau,1}=a_{\sigma\tau,1}$ for any $\sigma,\tau\in G$.
\end{lemma}

\begin{proof}
    Take $\tau=1$, then $\sigma a_{1,\gamma}=a_{\sigma,1}$, hence $a_{1,\gamma}=a_{1,1}$. Take $\gamma=1$, then $\sigma a_{\tau,1}=a_{\sigma\tau,1}$.
\end{proof}

\begin{proposition} \label{thm:cochain corresponds to lifting}
    If $a_{\sigma,\tau}$ induces $U/A\approx G$, then there exists $u_\sigma\in G$ such that $j(u_\sigma)=\sigma$ and $a_{\sigma,\tau}=u_\sigma u_\tau u_{\sigma\tau}^{-1}$.
\end{proposition}

\begin{proof}
    Let $u_\sigma=(x_\sigma,\sigma)$ for some $x_\sigma\in A$. Then we see \ali\[ u_\sigma u_\tau u_{\sigma\tau}^{-1}&=(x_\sigma+\sigma x_\tau+a_{\sigma,\tau},\sigma\tau)(-\tau^{-1
    }\sigma^{-1} x_{\sigma\tau}-a_{1,1}-a_{\tau^{-1}\sigma^{-1},\sigma\tau},\tau^{-1}\sigma^{-1}) \\ &=(x_\sigma+\sigma x_\tau+a_{\sigma,\tau}-x_{\sigma\tau}-\sigma\tau a_{\tau^{-1}\sigma^{-1},\sigma\tau}+a_{\sigma\tau,\tau^{-1}\sigma^{-1}}-\sigma\tau a_{1,1},1) \] Apply \cref{thm:condition of cochain to induce group extension} with $(\sigma,\tau,\gamma)\to(\sigma\tau,\tau^{-1}\sigma^{-1},\sigma\tau)$, we see \[ \sigma\tau a_{\tau^{-1}\sigma^{-1},\sigma\tau}=a_{\sigma\tau,\tau^{-1}\sigma^{-1}}-a_{\sigma\tau,1}+a_{1,\sigma\tau} \] Therefore, we have \ali\[ u_\sigma u_\tau u_{\sigma\tau}^{-1}&=(x_\sigma+\sigma x_\tau-x_{\sigma\tau}+a_{\sigma,\tau}+a_{\sigma\tau,1}-a_{1,\sigma\tau}-\sigma\tau a_{1,1},1) \\ &=(x_\sigma+\sigma x_\tau-x_{\sigma\tau}+a_{\sigma\tau,1}+a_{\sigma,\tau}-a_{1,1}-a_{\sigma\tau,1},1) \\ &=(x_\sigma+\sigma x_\tau-x_{\sigma\tau}+a_{\sigma,\tau}-a_{1,1},1) \] As we've remarked, the embedding $A\hookrightarrow U$ if given by $a\mapsto(a-a_{1,1},1)$, hence our goal is to find $x_\sigma$ such that $x_\sigma+\sigma x_\tau-x_{\sigma\tau}=0$. Take $x_\sigma=0$ and we finish the proof.
\end{proof}

\begin{definition}
    Two group extensions $U,U'$ of $G$-module $A$ are said to be \emph{isomorphic}, if there exists a group isomorphism $f:U_1\to U_2$ such that the following diagram is commutative:
    % https://q.uiver.app/#q=WzAsMTAsWzIsMCwiVSJdLFsxLDAsIkEiXSxbMCwwLCIwIl0sWzMsMCwiRyJdLFs0LDAsIjAiXSxbMCwxLCIwIl0sWzEsMSwiQSJdLFsyLDEsIlUnIl0sWzMsMSwiRyJdLFs0LDEsIjAiXSxbMiwxXSxbNSw2XSxbMSw2LCJcXGlkIiwyXSxbMSwwXSxbMCw3LCJmIiwyXSxbNiw3XSxbNyw4XSxbOCw5XSxbMCwzXSxbMyw0XSxbMyw4LCJcXGlkIiwyXV0=
    \[\begin{tikzcd}[row sep=2.25em]
        0 & A & U & G & 0 \\
        0 & A & {U'} & G & 0
        \arrow[from=1-1, to=1-2]
        \arrow[from=2-1, to=2-2]
        \arrow["\id"', from=1-2, to=2-2]
        \arrow[from=1-2, to=1-3]
        \arrow["f"', from=1-3, to=2-3]
        \arrow[from=2-2, to=2-3]
        \arrow[from=2-3, to=2-4]
        \arrow[from=2-4, to=2-5]
        \arrow[from=1-3, to=1-4]
        \arrow[from=1-4, to=1-5]
        \arrow["\id"', from=1-4, to=2-4]
    \end{tikzcd}\]
\end{definition}

\begin{analysis}
    Now we study on the problem of what $a_{\sigma,\tau}$ induces isomorphic group extensions. Let $f:U'\to U$ be an isomorphism of group extensions induced by $a'_{\sigma,\tau}$ and $a_{\sigma,\tau}$ respectively. We've already shown that there exists a lifting $u_\sigma\in U$ (and $u'_\sigma\in U'$ respectively) such that $a_{\sigma,\tau}=u_\sigma u_\tau u_{\sigma\tau}^{-1}$ (and $a'_{\sigma,\tau}=u'_\sigma u'_\tau u'^{-1}_{\sigma\tau}$ respectively). Since $j'\circ f=j$, we may write $f(u'_\sigma)=x_\sigma u_\sigma$ with $x_\sigma\in A$. Therefore, \[ a'_{\sigma,\tau}=f(a'_{\sigma,\tau})=f(u'_\sigma u'_\tau u'^{-1}_{\sigma\tau})=(x_\sigma,\sigma)(x_\tau,\tau)(x_{\sigma\tau},\sigma\tau)^{-1}=(x_\sigma+\sigma x_\tau-x_{\sigma\tau}+a_{\sigma,\tau}-a_{1,1},1)=x_\sigma x_\tau^\sigma x_{\sigma\tau}^{-1}a_{\sigma,\tau} \] Hence, two $a_{\sigma,\tau}$ induce ismorphic group extensions if and only if they differ by $(\sigma,\tau)\mapsto x_\sigma x_\tau^\sigma x_{\sigma\tau}^{-1}$.
\end{analysis}

\begin{analysis}
    Now we've already given the condition of $a_{\sigma,\tau}$ to induce a group extension, and also given the condition of when two induced group extensions are isomorphic. Let $Z$ be the set of $a_{\sigma,\tau}$ inducing a group extension, i.e., $C=\left\{(\sigma,\tau)\mapsto a_{\sigma,\tau}:\sigma a_{\tau,\gamma}=a_{\sigma,\tau}-a_{\sigma,\tau\gamma}+a_{\sigma\tau,\gamma}\right\}$. If both $a_{\sigma,\tau}$ and $a'_{\sigma,\tau}$ belongs to $C$, we see $a_{\sigma,\tau}+a'_{\sigma,\tau}$ and $-a_{\sigma,\tau}$ belongs to $C$. Hence $C$ has an abelian group structure. Moreover, two $a_{\sigma,\tau}$ induce isomorphic group extensions if and only if they differ by $(\sigma,\tau)\mapsto x_\sigma+\sigma x_\tau-x_{\sigma\tau}$. We denote by $B$ the set of such maps. It is easy to verify $B$ is a subgroup of $C$. Hence we have:
\end{analysis}

\begin{proposition}
    Isomorphism classes of group extensions corresponds one-to-one to elements of $C/B$.
\end{proposition}

\begin{analysis}
    Let $P_n=\Z[G^{n+1}]$, the free abelian group with basis $G^{n+1}$ equipped with the group action $s(g_0,\cdots,g_n)=(sg_0,\cdots,sg_n)$. Define a map $\varepsilon:P_0\to\Z$ by $g\mapsto0$ and $d:P_n\to P_{n-1}$ by \[ d(g_0,\cdots,g_n)=\sum_{i=0}^n{(-1)^i(g_0,\cdots,\hat{g_i},\cdots,g_n)} \] where $\hat{g_i}$ means excluding the term. We already know that \[ \cdots\to P_n\to\cdots\to P_1\to P_0\to\Z\to0\] is an exact sequence, hence it is a free resolution of $\Z$. We take the functor $\Hom_G(\cdot,A)$, and obtain a complex $K=\Hom_G(P,A)$. Therefore, $H^n(G,A)=H^n(K)$.

    Now given $f\in K^n=\Hom_G(\Z[G^{n+1}],A)$, we define a map $\varphi:G^n\to A$ by \[ \varphi(g_1,\cdots,g_n)=f(1,g_1,g_1g_2,\cdots,g_1\cdots g_n) \] Since $f$ is a $G$-homomorphism, it is uniquely determined by $\varphi$. By definition, \[ (df)(g_0,\cdots,g_{n+1})=\sum_{i=0}^{n+1}{f(g_0,\cdots,\hat{g_i},\cdots,g_{n+1})} \] Therefore, under its correspondence with $\varphi$, we have \[ (d\varphi)(g_1,\cdots,g_{n+1})=g_1\cdot\varphi(g_2,\cdots,g_{n+1})+\sum_{i=1}^n{(-1)^i\varphi(g_1,\cdots,g_ig_{i+1},\cdots,g_{n+1})}+(-1)^{n+1}\varphi(g_1,\cdots,g_n) \] We call $\varphi$ an \emph{$n$-cochain} (of $G$ in $A$). If $\varphi\in\ker d$, i.e., $d\varphi=0$, we call $\varphi$ an \emph{$n$-cocycle}, and if $\varphi\in\im d$, we call $\varphi$ an \emph{$n$-coboundary}. Thus the cohomology group is the quotient group of cocycles by coboundaries.

    Now we consider the case when $n=2$. The $2$-cochains are maps $\varphi:G\times G\to A$. \[ (d\varphi)(g_1,g_2,g_3)=g_1\cdot\varphi(g_2,g_3)-\varphi(g_1g_2,g_3)+\varphi(g_1,g_2g_3)-\varphi(g_1,g_2) \] A $2$-cocycle if a $2$-cochain $\varphi$ satisfying $d\varphi=0$, i.e., $g_1\cdot\varphi(g_2,g_3)=\varphi(g_1g_2,g_3)-\varphi(g_1,g_2g_3)+\varphi(g_1,g_2)$. Surprisingly, it coincides with the condition of $a_{\sigma,\tau}$ to induce a group extension. Moreover, let $\psi:G\to A$ be a $1$-cochain, and we have $$ (d\psi)(g_1,g_2)=g_1\cdot\psi(g_2)-\psi(g_1g_2)+\psi(g_1) $$ A $2$-coboundary is a $2$-cochain of the form $d\psi$, and it also coincides with the condition of $a_{\sigma,\tau}$ to induce the same group extension. Therefore, $a_{\sigma,\tau}$ is in fact a $2$-cochain: the condition of it to induce a group extension is to be a $2$-cocycle, and the condition of it to induce the same group extension is to differ by a $2$-coboundary. Hence, we have the following theorem:
\end{analysis}

\begin{theorem}
    Let $A$ be a $G$-module. Then isomorphism classes of group extensions of $A$ are in a natural one-to-one correspondence with the elements of the second cohomology group $H^2(G,A)$.
\end{theorem}

\begin{remark}
    Given a $2$-cocycle $a_{\sigma,\tau}$, we can construct a group extension $U=A\times G$ with $$ (a,\sigma)\cdot(b,\tau)=(a+b^\sigma+a_{\sigma,\tau},\sigma\tau) $$ However, this extension is not simple enough since the way $A$ is embedded into $U$ is by $a\mapsto(a-a_{1,1},1)$. Hence we wish $a_{1,1}=0$, which leads us to prove the following proposition:
\end{remark}

\begin{proposition} \label{thm:2-cocycle to be simple}
    Let $\alpha\in H^2(G,A)$. Then there exists a $2$-cocycle $a_{\sigma,\tau}$ of class $\alpha$ such that $a_{1,\sigma}=a_{\sigma,1}=0$.
\end{proposition}

\begin{proof}
    Let $b_{\sigma,\tau}$ be a $2$-cocycle of $\alpha$. We define $c_{\sigma,\tau}=\sigma b_{1,1}$. It is easy to verify that $c_{\sigma,\tau}$ is actually a $2$-coboundary. Hence $a_{\sigma,\tau}=b_{\sigma,\tau}-c_{\sigma,\tau}$ is also of class $\alpha$. $b_{1,\sigma}=b_{1,1}$ and $b_{\sigma,1}=\sigma b_{1,1}$, so proof.
\end{proof}

%%%%%%%%%%%%%%%%%%%
%%% NEW SECTION %%%
%%%%%%%%%%%%%%%%%%%

\sectione{Homomorphism of Group Extensions and Tranfer}

\begin{definition}
    Let $U/A\approx G$ and $U'/A'\approx G'$ be two group extensions, with two given group homomorphisms $f:A\to A'$ and $\varphi:G\to G'$, then a \emph{group extension homomorphism} is a group homomorphism $F:U\to U'$ such that the following diagram is commutative:
    % https://q.uiver.app/#q=WzAsNixbMCwwLCJBIl0sWzAsMSwiQSciXSxbMSwwLCJVIl0sWzEsMSwiVSciXSxbMiwwLCJHIl0sWzIsMSwiRyciXSxbMCwxLCJmIiwyXSxbMCwyXSxbMiw0XSxbMiwzLCJGIiwyXSxbMSwzXSxbNCw1LCJcXHZhcnBoaSIsMl0sWzMsNV1d
    \[\begin{tikzcd}[column sep=scriptsize,row sep=2.25em]
        A & U & G \\
        {A'} & {U'} & {G'}
        \arrow["f"', from=1-1, to=2-1]
        \arrow[from=1-1, to=1-2]
        \arrow[from=1-2, to=1-3]
        \arrow["F"', from=1-2, to=2-2]
        \arrow[from=2-1, to=2-2]
        \arrow["\varphi"', from=1-3, to=2-3]
        \arrow[from=2-2, to=2-3]
    \end{tikzcd}\]
\end{definition}

\begin{analysis}
    We wish to find a condition of $f$ and $\varphi$ being able to extend to $F$. First we take a $2$-cocycle of $U$ satisfying the condition of \cref{thm:2-cocycle to be simple}, say $a_{\sigma,\tau}$ and $a'_{\sigma,\tau}\in U'$ similarly. Firstly, $$ f(a^\sigma)=f(u_\sigma au_\sigma^{-1})=F(u_\sigma)f(a)F(u_\sigma)^{-1}=f(a)^{F(u_\sigma)}=f(a)^{\varphi(\sigma)} $$ Hence the $f$ is a $G$-homomorphism with $A'$ regarded as a $G$-module via $\varphi$. Secondly, we see $F(0,\sigma)=(x_\sigma,\varphi(\sigma))$
\end{analysis}
