\chaptere{Global Class Field Theory}

\sectione{Artin's Reciprocity Law}

\begin{notation}
    We denote by $K$ a global field and $\MMM_K$ its set of places. $S$ is often a finite set of places of $K$ containing all the archimedean places. We denote by $I^S$ the free abelian group generated by $\MMM_K-S$.

    $L$ is often a finite Galois extension of $K$, and in such cases, $S$ is often required to contain the ramified primes.
\end{notation}

\begin{analysis}
     Let $L/K$ be a finite Galois extension of global fields. For any unramified prime $\ppp$ in $K$, let $\PPP$ be a prime in $L$ above $\ppp$. Since $L_\PPP/K_\ppp$ is unramified, we see $\Gal(L_\PPP/K_\ppp)$ is cyclic of order $f(\PPP/\ppp)$, and we let $\sigma_\PPP$ be the Frobenius map in $\Gal(L_\PPP/K_\ppp)$. We know that the local Galois group can be embedded naturally into the global one, hence $\sigma_\PPP$ can be regarded as an element of $\Gal(L/K)$.

     Now let $\sigma\in\Gal(L/K)$, then $\PPP^\sigma$ is also a prime in $L$ above $\ppp$. We have a natural isomorphism between $\Gal(L_\PPP/K_\ppp)$ and $\Gal(L_{\PPP^\sigma}/K_\ppp)$ as subgroups of $\Gal(L/K)$ by mapping $\tau$ to $\sigma\tau\sigma^{-1}$. Thus $\sigma_{\PPP^\sigma}=\sigma\sigma_\PPP\sigma^{-1}$. Therefore, we see fixed a prime $\ppp$ in $K$, $\sigma_\PPP$ falls into the same conjugacy class in $\Gal(L/K)$ for all primes $\PPP$ above $\ppp$. Thus we can define the Artin map $\Art_K:\MMM_K-S\to[\Gal(L/K)]$ by mapping $\ppp$ to the conjugacy class of $\sigma_\PPP$. Since a conjugacy class is mapped to a single element in the abeianization, we induces $\Art_K:\MMM_K-S\to\Gal(L/K)^{ab}$. Since $I^S$ has $\MMM_K-S$ as its basis, we may extend $\Art_K$ to $\Art_K:I^S\to\Gal(L/K)^{ab}$.
\end{analysis}

\begin{notation}
    Let $K'/K$ be a subextension of $L/K$, and $S'\subseteq\MMM_{K'}$ be a finite set that only contains the primes above $S$. Then denote the norm map by $N_{K'/K}:I^{S'}\rightarrow I^S$, defined by $$ N_{K'/K}(\PPP)=\ppp^{f(\PPP/\ppp)}\text{ where }\PPP\text{ lies above }\ppp $$
\end{notation}

\begin{proposition}
    The following diagram is commutative:
        % https://q.uiver.app/#q=WzAsNCxbMSwxLCJcXEdhbChML0spXnthYn0iXSxbMSwwLCJcXEdhbChML0snKV57YWJ9Il0sWzAsMCwiSV57Uyd9Il0sWzAsMSwiSV5TIl0sWzIsMSwiXFxBcnRfe0snfSJdLFszLDAsIlxcQXJ0X0siXSxbMiwzLCJOX3tLJy9LfSIsMl0sWzEsMCwiXFx0aGV0YSJdXQ==
    \[\begin{tikzcd}[sep=large]
        {I^{S'}} & {\Gal(L/K')^{ab}} \\
        {I^S} & {\Gal(L/K)^{ab}}
        \arrow["{\Art_{K'}}", from=1-1, to=1-2]
        \arrow["{\Art_K}", from=2-1, to=2-2]
        \arrow["{N_{K'/K}}"', from=1-1, to=2-1]
        \arrow["\theta", from=1-2, to=2-2]
    \end{tikzcd}\]
    where $\theta$ is induced by the inclusion map $\Gal(L/K)\rightarrow\Gal(L/K')$.
\end{proposition}

\end{document}
