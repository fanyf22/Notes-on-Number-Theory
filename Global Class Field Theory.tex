\chaptere{Global Class Field Theory}

%%%%%%%%%%%%%%%%%%%
%%% NEW SECTION %%%
%%%%%%%%%%%%%%%%%%%

\sectione{Artin Map and Reciprocity Law}

\begin{notation}
    We denote by $K$ a global field and $\MMM_K$ its set of places. $S$ is often a finite set of places of $K$ containing all the archimedean places. We denote by $I^S$ the free abelian group generated by $\MMM_K-S$.

    $L$ is often a finite Galois extension of $K$, and in such cases, $S$ is often required to contain the ramified primes.
\end{notation}

\begin{analysis}
     Let $L/K$ be a finite Galois extension of global fields. For any unramified prime $\ppp$ in $K$, let $\PPP$ be a prime in $L$ above $\ppp$. Since $L_\PPP/K_\ppp$ is unramified, we see $\Gal(L_\PPP/K_\ppp)$ is cyclic of order $f(\PPP/\ppp)$, and we let $\sigma_\PPP$ be the Frobenius map in $\Gal(L_\PPP/K_\ppp)$. We know that the local Galois group can be embedded naturally into the global one, hence $\sigma_\PPP$ can be regarded as an element of $\Gal(L/K)$.

     Now let $\sigma\in\Gal(L/K)$, then $\PPP^\sigma$ is also a prime in $L$ above $\ppp$. We have a natural isomorphism between $\Gal(L_\PPP/K_\ppp)$ and $\Gal(L_{\PPP^\sigma}/K_\ppp)$ as subgroups of $\Gal(L/K)$ by mapping $\tau$ to $\sigma\tau\sigma^{-1}$. Thus $\sigma_{\PPP^\sigma}=\sigma\sigma_\PPP\sigma^{-1}$. Therefore, we see fixed a prime $\ppp$ in $K$, $\sigma_\PPP$ falls into the same conjugacy class in $\Gal(L/K)$ for all primes $\PPP$ above $\ppp$. Thus we can define the map $F_{L/K}:\MMM_K-S\to[\Gal(L/K)]$ by mapping $\ppp$ to the conjugacy class of $\sigma_\PPP$. Since a conjugacy class is mapped to a single element in the abelianization, we induces $\Art_{L/K}:\MMM_K-S\to\Gal(L/K)^{ab}$. By definition of $I^S$, we may extend to the \emph{Artin map} $\Art_{L/K}:I^S\to\Gal(L/K)^{ab}$.
\end{analysis}

\begin{proposition}
    Let $L'/K'$ be a Galois extension such that $K\subseteq K'$ and $L\subseteq L'$. Let $S'$ be a finite set of places of $K'$ containing all the archimedean places, the primes ramified in $L'$ and all primes above $S$. Then we have
    % https://q.uiver.app/#q=WzAsNCxbMSwxLCJcXEdhbChML0spXnthYn0iXSxbMSwwLCJcXEdhbChMJy9LJylee2FifSJdLFswLDAsIklee1MnfSJdLFswLDEsIkleUyJdLFsyLDEsIlxcQXJ0X3tLJ30iXSxbMywwLCJcXEFydF9LIl0sWzIsMywiTl97SycvS30iLDJdLFsxLDAsIlxcdGhldGEiXV0=
    \[\begin{tikzcd}[sep=large]
        {I^{S'}} & {\Gal(L'/K')^{ab}} \\
        {I^S} & {\Gal(L/K)^{ab}}
        \arrow["{\Art_{K'}}", from=1-1, to=1-2]
        \arrow["{\Art_{L/K}}", from=2-1, to=2-2]
        \arrow["{N_{K'/K}}"', from=1-1, to=2-1]
        \arrow["\theta", from=1-2, to=2-2]
    \end{tikzcd}\]
    where $\theta$ is induced by the restriction map $\Gal(L'/K')\rightarrow\Gal(L/K)$.
\end{proposition}

\begin{proof}
    Let $v'\in\MMM_{K'}-S'$, and $v\in\MMM_K$ below $v'$. Let $w'$ be a place in $L'$ above $v'$, and $w$ be the place in $L$ below $w'$. Thus $\Art_{L/K}(v)$ (and $\Art_{L'/K'}(v')$ respectively) is the Frobenius map $\sigma_w$ of $L_w/K_v$ (and $\sigma_{w'}$ of $L_{w'}/K_{v'}$ respectively). We see the Frobenius map is a power map of the cardinality of the residue field of the ground field, hence we have $\sigma_{w'}=\sigma_w^{[\kappa(v'):\kappa(v)]}=\sigma_w^{f(v'/v)}$. We see $N_{K'/K}v'=f(v'/v)\cdot v$, so proof.
\end{proof}

\begin{notation}
    For $a\in K^\times$, we denote by $(a)^S$ the element in $I^S$: \dis$(a)^S=\sum_{v\not\in S}{v(a)\cdot v}$
\end{notation}

\begin{theorem}[Reciprocity Law in the Crude Form]
    Let $L/K$ be a finite abelian extension of global fields, and $S$ be a set of places in $K$ consisting of the archimedean ones and those ramified in $L$. Then there exists $\varepsilon>0$ such that if $a\in K^\times$ and $|a-1|_v<\varepsilon$ for all $v\in S$, then $\Art_{L/K}((a)^S)=1$.
\end{theorem}

\begin{definition}
    Let $K$ be a global field and $S\subseteq\MMM_K$ consisting of all archimedean places, and let $G$ be a abelian topological group. Then a homomorphism $\phi:I^S\to G$ is said to be \emph{admissible} if for each open neighbourhood $N$ of the identity element $1$ of $G$, there exists $\varepsilon>0$ such that $\phi((a)^S)\in N$ whenever $a\in K^\times$ and $|a-1|_v<\varepsilon$ for all $v\in S$.
\end{definition}

\begin{remark}
    When $G$ is discrete, we simply take $N=1$, thus we have:
\end{remark}

\begin{theorem}[Reciprocity Law]
    Let $L/K$ be an abelian extension of global fields, then $\Art_{L/K}$ is admissible.
\end{theorem}

%%%%%%%%%%%%%%%%%%%
%%% NEW SECTION %%%
%%%%%%%%%%%%%%%%%%%

\sectione{Chevalley's Interpretation by Id\`eles}

\begin{notation}
    We first review the notations of id\`eles. Let $K$ be a global field, and $S$ be any finite subset of $\MMM_K$ consisting of achimedean places, we denote by \dis$J_{K,S}=\prod_{v\in S}{K_v^\times}\times\prod_{v\not\in S}{U_v}$, equipped with the product topology. For $S\subseteq S'$, we have a natural continuous homomorphism $J_{K,S}\to J_{K,S'}$, and it induces a direct system $(J_{K,S})_S$. We define the group of id\`eles to be \dis$J_K=\varinjlim_S{J_{K,S}}$, and denote by $J_K^S$ the id\`eles that have coordinate $1$ at $S$.
\end{notation}

\begin{notation}
    Let $x=(x_v)$ be an id\`ele, we denote by \dis$(x)^S=\sum_{v\not\in S}{v(a_v)\cdot v}$.
\end{notation}

\begin{proposition}
    Let $G$ be a complete abelian topological group and $\phi:I^S\to G$ admissible. Then there exists a unique continuous homomorphism $\psi:J_K\to G$ such that
    \begin{multicols}{2}
        \begin{enumerate}
            \item $\psi(K^\times)=1$;
            \item $\psi(x)=\phi((x)^S)$ for $x\in J_K^S$.
        \end{enumerate}
    \end{multicols}
    Conversely, if $\psi$ is a continuous homomorphism $J_K\to G$ such that $\psi(K^\times)=1$, then $\psi$ comes from some admissible pair $(\phi,S)$ as defined above, provided there exists an open neighbourhood of $1$ in $G$ in which $0$ is the only subgroup.
\end{proposition}

% TODO: proof

\begin{remark}
    It is clear that if such a $\psi$ exists for a given $\phi$ and $S$, then it induces a continuous homomorphism of the id\`ele class group $C_K\approx J_K/K^\times$ to $G$. We also denote by $\phi$ ths induced homomorphism.
\end{remark}

\begin{remark}
    If $\phi$ and $S$ induce such a homomorphism $\psi$, then $\phi$ and $S'$ also induce a homomorphism $\psi'$ for any $S'\supseteq S$. By the uniqueness, we have $\psi=\psi'$. In particular, if two $\phi$'s on $I^S$ coincide with $I^{S'}$ for some finite $S'\supseteq S$, they are actually equal on $I^S$.
\end{remark}

\begin{corollary}
    The reciprocity law holds for a finite abelian extension $L/K$ of global fields, if and only if there exists a continuous homomorphism $\psi:J_K\to\Gal(L/K)^{ab}$ such that $\psi(K^\times)=1$ and $\psi(x)=\Art_{L/K}((x)^S)$ for $x\in J_K^S$, where $S$ consists of the archimedean places and the primes ramified in $L$.
\end{corollary}

\begin{notation}
    Let $L/K$ be a finite extension of global fields, then for each $w\in\MMM_L$ and $v\in\MMM_K$ below $w$, $L_w/K_v$ is a finite extension of local fields. Let $a_w\in L_w$, then we have the local norm $N_{L_w/K_v}a_w$. For any $(a_w)\in J_L$, we define its norm to be given by \dis$(N_{L/K}(a_w))_v=\prod_{w/v}{N_{L_w/K_v}{a_w}}\in J_K$.
\end{notation}

\begin{proposition}
    If the reciprocity law holds for $L/K$ and $L'/K'$, then the following diagram is commutative:
    % https://q.uiver.app/#q=WzAsNCxbMCwwLCJKX3tLJ30iXSxbMCwxLCJKX0siXSxbMSwwLCJcXEdhbChMJy9LJylee2FifSJdLFsxLDEsIlxcR2FsKEwvSylee2FifSJdLFswLDIsIlxccHNpX3tMJy9LJ30iXSxbMCwxLCJOX3tML0t9IiwyXSxbMSwzLCJcXHBzaV97TC9LfSJdLFsyLDMsIlxcdGhldGEiXV0=
    \[\begin{tikzcd}[sep=large]
        {J_{K'}} & {\Gal(L'/K')^{ab}} \\
        {J_K} & {\Gal(L/K)^{ab}}
        \arrow["{\psi_{L'/K'}}", from=1-1, to=1-2]
        \arrow["{N_{L/K}}"', from=1-1, to=2-1]
        \arrow["{\psi_{L/K}}", from=2-1, to=2-2]
        \arrow["\theta", from=1-2, to=2-2]
    \end{tikzcd}\]
\end{proposition}

% TODO: proof

\begin{corollary}
    If the reciprocity law holds for $L/K$, then $\psi_{L/K}(N_{L/K}J_L)=1$.
\end{corollary}

%%%%%%%%%%%%%%%%%%%
%%% NEW SECTION %%%
%%%%%%%%%%%%%%%%%%%

\sectione{Statements of Main Theorems on Abelian Extensions}


